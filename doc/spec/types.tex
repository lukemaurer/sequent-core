\documentclass{article}

\usepackage{amsmath}
\usepackage{amssymb}
\usepackage{proof}
\usepackage{hyperref}
\usepackage{preamble}

\begin{document}

\title{Typing Dual System FC and Sequent Core}

\maketitle

\section{Syntax}
\label{sec:syntax}

\begin{figure}[h]
\centering
\begin{gather*}
\begin{aligned}
  x, y, z, f, g, h, K &\in Var
  \\
  q, r &\in KontVar
  \\
  a, b, T &\in TypeVar
  \\
  \com &\in Command
  &
  &::= \mcut{\tm}{\ko}{\binds}
  \\
  \tm &\in Term
  &
  &::= x
  \Alt \fn{\ann x \ty} \tm
  \Alt \compute{\ann q \ty} \com
  \Alt \lit
  \Alt \ty
  \Alt \cn
  \Alt \paks{\vect[i]{\ty_i}}{\vect[j]{\tm_j}}
  \\
  \ko &\in Kont
  &
  &::= q
  \Alt \app{\tm}{\ko}
  \Alt \koerce{\cn}{\ko}
  \Alt \caseas{\ann x \ty}{\alts}
  \Alt \fnk{\paks{\vect[i]{\ann{a_i}{\ki_i}}}{\vect[j]{\ann{x_j}{\ty_j}}}}{\com}
  \\
  \binds &\in Bindings
  &
  &::= \vect[i]{\abind_i}
  \\
  \abind &\in Binding
  &
  &::= \bind{\ann x \ty}{\tm}
  \Alt \Rec \vect[i]{\bind{\ann{x_i}{\ty_i}}{\tm_i}}
  \Alt \bind{\ann q \sty}{\ko}
  \Alt \Rec \vect[i]{\bind{\ann{q_i}{\sty_i}}{\ko_i}}
  \\
  \alts &\in Alternatives
  &
  &::= \vect[i]{\analt_i}
  \\
  \analt &\in Alternative
  &
  &::= \alt{\blank}{\com}
  \Alt \alt{K ~ \vect[i]{\ann{b_i}{\ki_i}} ~ \vect[j]{\ann{x_j}{\ty_j}}}{\com}
  \Alt \alt{\lit}{\com}
  \\
  \ty &\in Type
  &
  &::= \dots
  \\
  \sty &\in SequentType
  &
  &::= \ty
  \Alt \stackty{\vect[i]{a_i:\ki_i}}{\vect[j]{\ty_j}}
  \\
  \ki &\in Kind
  &
  &::= \dots
  \\
  \cn &\in Coercion
  &
  &::= \dots
  \\
  \decls &\in Declarations
  &
  &::= \vect[i]{\decl_i}
  \\
  \decl &\in Declaration
  &
  &::= \dots
  \\
  \pgm &\in Program
  &
  &::= \decls; \com
\end{aligned}
\end{gather*}
\caption{Syntax of Dual System FC}
\label{fig:dual-fc-syntax}
\end{figure}

The syntax for Dual System FC are shown in Figure~\ref{fig:dual-fc-syntax}.
Types, kinds, coercions, and declarations are unchanged by the sequent calculus
representation, so they are elided here.  Note that data constructors, written
$K$, are treated as a special sort of variable in the syntax, and additionally
type constructors, written $T$, are treated as a special sort of type variable.
We use some conventional shorthands to make programs easier to read:
\begin{itemize}
\item If the type annotations on variables, \emph{ie} the $\ty$ in $\ann x \ty$,
  are not important to a particular example, we will often omit them.
\item The function call constructor $\app{\blank}{\blank}$ associates to the
  right, so $\app{1}{\app{2}{\app{3}{q}}}$ is the same as
  $\app{1}{(\app{2}{(\app{3}{q})})}$.  Similarly, coercion continuations
  associate to the right as well, so that $\koerce{\cn_1}{\koerce{\cn_2}{q}}$ is
  the same as $\koerce{\cn_1}{(\koerce{\cn_2}{q})}$.  Both function calls and
  coercions share the same precedence and may be intermixed, so that
  $\app{1}{\koerce{\cn_2}{\app{3}{q}}}$ is the same as
  $\app{1}{(\koerce{\cn_2}{(\app{3}{q})})}$.
\item If a command does not have any associated bindings with it, so that
  $\binds$ is empty in $\mcut{\binds}{\tm}{\ko}$, then we will omit the $\Let$
  form altogether and just write $\cut{\tm}{\ko}$.
\item We will not always write the binding variable $\ann x \ty$ in the case
  continuation $\caseas{\ann x \ty}{\alts}$ when it turns out that $x$ is never
  referenced in $\alts$ or $\alts$, instead writing $\Case\alts$.  If instead
  $\ann x \ty$ is only referenced in the default alternative
  $\alt{\blank}{\com}$ in $\alts$, we will prefer to write $\ann x \ty$ in place
  of the wildcard $\blank$ pattern.  This often arises in a case continuation
  with \emph{only} a default alternative,
  $\caseas{\ann x \ty}{\alt{\blank}{\com}}$, which we write as the shortened
  $\Case\alt{\ann x \ty}{\com}$.
\end{itemize}

\section{Scope and exit analysis}
\label{sec:scope-analysis}

\begin{figure}[h]
\centering
\begin{gather*}
\begin{aligned}
  \Gamma \in& Environment
  &
  &::= \mt
  \Alt \Gamma, x
  \Alt \Gamma, a
  \\
  \Delta \in& KoEnvironment
  &
  &::= \mt
  \Alt \Delta, q
\end{aligned}
\end{gather*}

Term scoping: $\Gamma \vdash \tm \Ok$
\begin{gather*}
  \infer
  {\Gamma \vdash x \Ok}
  {x \in \Gamma}
  \qquad
  \axiom{\Gamma \vdash \lit \Ok}
  \qquad
  \infer
  {\Gamma \vdash \compute q \com \Ok}
  {
    \Gamma; q \vdash \com \Ok
  }
  \\\\
  \infer
  {\Gamma \vdash \fn x \tm \Ok}
  {
    \Gamma, x \vdash \tm \Ok
  }
  \qquad
  \infer
  {\Gamma \vdash \fn a \tm \Ok}
  {
    \Gamma, a \vdash \tm \Ok
  }
  \qquad
  \infer
  {\Gamma \vdash \paks{\vect[i]{\ty_i}}{\vect[j]{\tm_j}} \Ok}
  {
    \vect[i]{\Gamma \vdash \ty_i \Ok}
    &
    \vect[j]{\Gamma \vdash \tm_j \Ok}
  }
\end{gather*}

Continuation scoping: $\Gamma; \Delta \vdash \ko \Ok$
\begin{gather*}
  \infer
  {\Gamma;\Delta \vdash q \Ok}
  {q \in \Delta}
  \qquad
  \infer
  {\Gamma;\Delta \vdash \app{\tm}{\ko} \Ok}
  {
    \Gamma \vdash \tm \Ok
    &
    \Gamma;\Delta \vdash \ko \Ok
  }
  \qquad
  \infer
  {\Gamma;\Delta \vdash \app{\ty}{\ko} \Ok}
  {
    \Gamma \vdash \ty \Ok
    &
    \Gamma;\Delta \vdash \ko \Ok
  }
  \\\\
  \infer
  {\Gamma;\Delta \vdash \koerce{\cn}{\ko} \Ok}
  {
    \Gamma \vdash \cn \Ok
    &
    \Gamma;\Delta \vdash \ko \Ok
  }
  \qquad
  \infer
  {\Gamma;\Delta \vdash \caseas{x}{\alts} \Ok}
  {
    \Gamma, x;\Delta \vdash \alts \Ok
  }
  \qquad
  \infer
  {\Gamma;\Delta \vdash \fnk{\paks{\vect[i]{a_i}}{\vect[j]{x_j}}} \com \Ok}
  {
    \Gamma,\vect[i]{a_i},\vect[j]{x_j};\Delta \vdash \com \Ok
  }
\end{gather*}

Command scoping: $\Gamma; \Delta \vdash \com \Ok$
\begin{gather*}
  \infer
  {\Gamma;\Delta \vdash \mcut{\tm}{\ko}{\binds} \Ok}
  {
    \Gamma;\Delta \vdash \binds : \Gamma';\Delta'
    &
    \Gamma,\Gamma' \vdash \tm \Ok
    &
    \Gamma,\Gamma'; \Delta,\Delta' \vdash \ko \Ok
  }
\end{gather*}


\emph{Further rules for $\Gamma \vdash \ty \Ok$ and
  $\Gamma \vdash \ki \Ok$, $\Gamma \vdash \cn \Ok$}
\caption{Scope and exit analysis for terms, continuations, and commands}
\label{fig:scoping-rules}
\end{figure}

\begin{figure}[h]
\centering

Binding and alternative scoping:
$\Gamma; \Delta \vdash \abind : \Gamma'; \Delta'$ and
$\Gamma; \Delta \vdash \analt \Ok$
\begin{gather*}
  \axiom{\Gamma;\Delta \vdash \mt : \mt;\mt}
  \qquad
  \infer
  {\Gamma;\Delta \vdash \binds; \abind : \Gamma'';\Delta''}
  {
    \Gamma;\Delta \vdash \binds : \Gamma';\Delta'
    &
    \Gamma';\Delta' \vdash \abind : \Gamma'';\Delta''
  }
\end{gather*}
\begin{gather*}
  \infer
  {\Gamma;\Delta \vdash \bind{x}{\tm} : \Gamma,x;\Delta}
  {\Gamma \vdash \tm \Ok}
  \qquad
  \infer
  {
    \Gamma;\Delta
    \vdash
    \Rec\vect[i]{\bind{x_i}{\tm_i}}
    :
    \Gamma,\vect[i]{x_i};\Delta
  }
  {
    \vect[i]
    {
      \Gamma, \vect[j]{x_j} \vdash \tm_i \Ok
    }
  }
  \\\\
  \infer
  {\Gamma;\Delta \vdash \bind{q}{\ko} : \Gamma;\Delta,q}
  {\Gamma;\Delta \vdash \ko \Ok}
  \qquad
  \infer
  {
    \Gamma;\Delta
    \vdash
    \Rec\vect[i]{\bind{q_i}{\ko_i}}
    :
    \Gamma;\Delta,\vect[i]{q_i}
  }
  {
    \vect[i]
    {
      \Gamma;\Delta,\vect[j]{q_j} \vdash \ko_i \Ok
    }
  }  
\end{gather*}
\begin{gather*}
  \infer
  {\Gamma;\Delta \vdash \vect[i]{\analt_i} \Ok}
  {
    \vect[i]{\Gamma;\Delta \vdash \analt_i \Ok}
  }
  \qquad
  \infer
  {\Gamma;\Delta \vdash \alt{\blank}{\com} \Ok}
  {\Gamma;\Delta \vdash \com \Ok}
  \qquad
  \infer
  {\Gamma;\Delta \vdash \alt{K ~ \vect[i]{b_i} \vect[i]{x_i}}{\com} \Ok}
  {\Gamma,\vect[i]{b_i},\vect[i]{x_i};\Delta \vdash \com \Ok}
  \qquad
  \infer
  {\Gamma;\Delta \vdash \alt{\lit}{\com} \Ok}
  {\Gamma;\Delta \vdash \com \Ok}
\end{gather*}

Program scoping: $\Gamma;\Delta \vdash \pgm \Ok$
\begin{gather*}
  \infer
  {\Gamma;\Delta \vdash \decls; c \Ok}
  {
    \Gamma \vdash \decls: \Gamma'
    &
    \Gamma';\Delta \vdash c \Ok
  }
\end{gather*}

\emph{Further rules for $\Gamma \vdash \decls : \Gamma'$ and
  $\Gamma \vdash \decl : \Gamma'$.}
\caption{Scope and exit analysis for bindings, alternatives, and
  programs}
\label{fig:scoping-rules-binds}
\end{figure}

The scoping rules for variables are shown in Figures~\ref{fig:scoping-rules} and
\ref{fig:scoping-rules-binds}, where the rules for scoping inside types, kinds,
coercions, and declarations are elided.  Continuation variables are treated
differently from the other sorts of variables, being placed in a separate
environment $\Delta$, in order to prevent non-functional uses of control flow.

Besides the normal rules for checking variable scope, these rules effectively
also perform an \emph{exit analysis} on a program (bindings, terms, commands,
\emph{etc}).  The one major restriction that we enforce is that terms must
always have a \emph{unique} exit point and cannot jump outside their scope.  The
intuition is:
\begin{quote}
  Terms cannot contain any references to free continuation variables.
\end{quote}
This restriction makes sure that $\lambda$-abstractions cannot close over
continuation variables available from its context, so that bound continuations
do not escape through a returned $\lambda$-abstraction.  Additionally, in all
computations $\comp{r}\com$, the underlying command $c$ has precisely one unique
exit point, namely $r$, which names the result of the computation.

If the command $c$ inside the well-scoped term $\comp{r}\com$ stops execution
with some value $V$ sent to some continuation variable $q$, then we know that:
\begin{itemize}
\item $q$ must be equal to $r$, due to the fact that $r$ is the only allowable
  free continuation variable inside of $\com$, and
\item $r$ does not appear free inside the resulting value $V$, again due to the
  scoping rules for continuation variables inside of a command.
\end{itemize}
In the simple case, this means execution of the term $\comp{r}\com$ yields
$\comp{r}\cut{V}{r}$, which $\eta$-reduces to just the value $V$ by the
previously mentioned reasoning.  Thus, evaluating a term always results in a
unique value.

Notice that these scoping rules, while not very complex, still manage to tell us
something about the expressive capabilities of the language.  For example,
recursive bindings can be between only terms or only continuations.  But why not
allow for is mutual recursion between both terms and continuations in the same
binding block?  It turns out that these scoping rules disallow any sort of
interesting mutual recursion between terms and continuations because terms are
\emph{prevented} from referencing continuations within their surrounding (or
same) binding environment.

For example, in a simple case where we have the recursive bindings:
\begin{gather*}
  \Rec \{ \bind{f}{\fn x v}; \bind{q}{\Case\alt{y}{c}} \}
\end{gather*}
then by the scoping rules, $q$ may call $f$ through $c$, but $f$ cannot jump
back to $q$ in $v$ because $\fn x v$ cannot contain the free reference to $q$.
Therefore, since there is no true mutual recursion between both $f$ and $q$, we
can break the recursive bindings into two separate blocks with the correct
scope:
\begin{gather*}
  \Rec \{ \bind{f}{\fn x v} \}; \Rec \{ \bind{q}{\Case\alt{y}{c}} \}
\end{gather*}
So this limitation results in no loss of expressiveness.  Indeed, we could
further the partitions into
\begin{enumerate}
\item first, the list of term bindings, and
\item second, the list of continuation bindings,
\end{enumerate}
since continuations can refer to previously bound terms but not vice versa.
However, we do not make this distinction here.

\section{Type checking}
\label{sec:typing}

\begin{figure}[h]
\centering
\begin{align*}
  \Gamma \in& Environment
  &
  &::= \mt
  \Alt \Gamma, x : \ty
  \Alt \Gamma, a : \ki
  \\
  \Delta \in& KoEnvironment
  &
  &::= \mt
  \Alt \Delta, q : \sty
\end{align*}

Term typing: $\Gamma \vdash \tm : \sty$
\begin{gather*}
  \infer
  {\Gamma \vdash x : \ty}
  {
    x:\ty \in \Gamma
  }
  \qquad
  \infer
  {\Gamma \vdash \lit : \ty}
  {\ty = literalType(\lit)}
  \qquad
  \infer
  {\Gamma \vdash \compute{\ann q \ty} \com : \ty}
  {
    \Gamma; q:\ty \vdash \com \Ok
  }
  \\\\
  \infer
  {\Gamma \vdash \fn{\ann x {\ty_1}} \tm : \ty_1 \to \ty_2}
  {
    \Gamma, x : \ty_1 \vdash \tm : \ty_2
  }
  \qquad
  \infer
  {\Gamma \vdash \fn{\ann a \ki} \tm : \forall \ann a \ki. \ty}
  {
    \Gamma, a:\ki \vdash \tm : \ty
  }
  \qquad
  \infer
  {
    \Gamma
    \vdash
    \paks{\vect[i]{\ty_i}}{\vect[j]{\tm_j}}
    :
    \stackty{\vect[i]{a_i:\ki_i}}{\vect[j]{\ty'_j}}
  }
  {
    \vect[i]{
      \Gamma \vdash \ty_i : \ki_i
    }
    &
    \vect[j]{
      \Gamma \vdash \tm_j : \ty'_j[\vect[i]{\ty_i/a_i}]
    }
  }
\end{gather*}

Continuation typing: $\Gamma; \Delta \vdash \ko : \sty$
\begin{gather*}
  \infer
  {\Gamma;\Delta \vdash q : \sty}
  {
    q : \sty \in \Delta
  }
  \qquad
  \infer
  {\Gamma;\Delta \vdash \app{\tm}{\ko} : \ty_1 \to \ty_2}
  {
    \Gamma \vdash \tm : \ty_1
    &
    \Gamma;\Delta \vdash \ko : \ty_2
  }
  \qquad
  \infer
  {\Gamma;\Delta \vdash \app{\ty_1}{\ko} : \forall \ann a \ki. \ty_2}
  {
    \Gamma \vdash \ty_1 : \ki
    &
    \Gamma;\Delta \vdash \ko : \ty_2[\ty_1/a]
  }
  \\\\
  \infer
  {\Gamma;\Delta \vdash \koerce{\cn}{\ko} : \ty_1}
  {
    \Gamma \vdash \cn : \eqty{\ty_1}{\ty_2}
    &
    \Gamma;\Delta \vdash \ko : \ty_2
  }
  \qquad
  \infer
  {\Gamma;\Delta \vdash \caseas{\ann x \ty}{\alts} : \ty}
  {
    \Gamma, x:\ty;\Delta \vdash \alts : \ty
  }
  \\\\
  \infer
  {
    \Gamma;\Delta
    \vdash
    \fnk{\paks{\vect[i]{a_i:\ki_i}}{\vect[j]{x_j:\ty_j}}} \com
    :
    \stackty{\vect[i]{a_i:\ki_i}}{\vect[j]{\ty_j}}
  }
  {
    \Gamma,\vect[i]{a_i:\ki_i},\vect[j]{x_j:\ty_j};\Delta \vdash \com \Ok
  }
\end{gather*}

Command typing: $\Gamma; \Delta \vdash \com \Ok$
\begin{gather*}
  \infer
  {\Gamma;\Delta \vdash \mcut{\tm}{\ko}{\binds} \Ok}
  {
    \Gamma;\Delta \vdash \binds : \Gamma';\Delta'
    &
    \Gamma,\Gamma' \vdash \tm : \sty
    &
    \Gamma,\Gamma' \vdash \sty : \star
    &
    \Gamma,\Gamma'; \Delta,\Delta' \vdash \ko : \sty
  }
\end{gather*}

\emph{Further rules for $\Gamma \vdash \ty : \ki$,
  $\Gamma \vdash \ki : \delta$, and
  $\Gamma \vdash \cn : \eqty{\ty_1}{\ty_2}$.}
\caption{Type checking rules for terms, continuations, and commands}
\label{fig:typing-rules}
\end{figure}

\begin{figure}[h]
\centering
Sequent type kinding: $\Gamma \vdash \sty : \ki$
\begin{gather*}
  \infer
  {\Gamma \vdash \stackty{\vect[i]{a_i:\ki_i}}{\vect[j]{\ty_j}} : \star}
  {
    \vect[j]{
      \Gamma, \vect[i]{a_i:\ki_i} \vdash \ty_j : \star
    }
  }
\end{gather*}

Binding typing: $\Gamma; \Delta \vdash \binds : \Gamma'; \Delta'$ and
$\Gamma; \Delta \vdash \abind : \Gamma'; \Delta'$
\begin{gather*}
  \axiom{\Gamma;\Delta \vdash \mt : \mt;\mt}
  \qquad
  \infer
  {\Gamma;\Delta \vdash \binds; \abind : \Gamma'';\Delta''}
  {
    \Gamma;\Delta \vdash \binds : \Gamma';\Delta'
    &
    \Gamma';\Delta' \vdash \abind : \Gamma'';\Delta''
  }
\end{gather*}
\begin{gather*}
  \infer
  {\Gamma;\Delta \vdash \bind{\ann x \ty}{\tm} : \Gamma,x:\ty;\Delta}
  {\Gamma \vdash \tm : \ty}
  \qquad
  \infer
  {
    \Gamma;\Delta
    \vdash
    \Rec\vect[i]{\bind{\ann{x_i}{\ty_i}}{\tm_i}}
    :
    \Gamma,\vect[i]{x_i:\ty_i};\Delta
  }
  {
    \vect[i]
    {
      \Gamma, \vect[j]{x_j:\ty_j} \vdash \tm_i : \ty_i
    }
  }
  \\\\
  \infer
  {\Gamma;\Delta \vdash \bind{\ann q \sty}{\ko} : \Gamma;\Delta,q:\sty}
  {\Gamma;\Delta \vdash \ko : \sty}
  \qquad
  \infer
  {
    \Gamma;\Delta
    \vdash
    \Rec\vect[i]{\bind{\ann{q_i}{\sty_i}}{\ko_i}}
    :
    \Gamma;\Delta,\vect[i]{q_i:\sty_i}
  }
  {
    \vect[i]
    {
      \Gamma;\Delta,\vect[j]{q_j:\sty_i} \vdash \ko_i : \sty_i
    }
  }  
\end{gather*}

Alternative typing: $\Gamma; \Delta \vdash \analt : \ty$
\begin{gather*}
  \infer
  {\Gamma;\Delta \vdash \vect[i]{\analt_i} : \ty}
  {
    \vect[i]{\Gamma;\Delta \vdash \analt_i : \ty}
  }
  \qquad
  \infer
  {\Gamma;\Delta \vdash \alt{\blank}{\com} : \ty}
  {\Gamma;\Delta \vdash \com \Ok}
  \qquad
  \infer
  {\Gamma;\Delta \vdash \alt{\lit}{\com} : \ty}
  {
    \Gamma \vdash \lit : \ty
    &
    \Gamma;\Delta \vdash \com \Ok
  }
  \\\\
  \infer
  {
    \Gamma;\Delta
    \vdash
    \alt{K ~ \vect[i]{\ann{b_{j'}}{\theta(\ki'_{j'})}} \vect[i]{\ann{x_{i}}{\theta(\ty_{i})}}}{\com}
    :
    T ~ \vect[j]{\tau_j}
  }
  {
    K
    :
    \vect[j]{\forall\ann{a_j}{\ki_j}.} \vect[j']{\forall\ann{b_{j'}}{\ki'_{j'}}.}
      \vect[i]{\ty_{i} \to} T ~ \vect[j]{a_j}
    \in
    \Gamma
    &
    \theta = [\vect[j]{\tau_j/a_j}]
    &
    \Gamma,
    \vect[j']{b_{j'}:\theta(\ki'_{j'})},
    \vect[i]{x_{i}:\theta(\ty_{i})};
    \Delta
    \vdash
    \com \Ok
  }
\end{gather*}

Program typing: $\Gamma;\Delta \vdash \pgm \Ok$
\begin{gather*}
  \infer
  {\Gamma; \Delta \vdash \decls; c \Ok}
  {
    \Gamma \vdash \decls : \Gamma'
    &
    \Gamma';\Delta \vdash c \Ok
  }
\end{gather*}

\emph{Further rules for $\Gamma \vdash \decls : \Gamma'$ and
  $\Gamma \vdash \decl : \Gamma'$.}
\caption{Type checking rules for bindings, alternatives, and programs}
\label{fig:typing-rules-binds}
\end{figure}

The typing rules for Dual System FC are given in Figures~\ref{fig:typing-rules}
and \ref{fig:typing-rules-binds}.  The type of a term classifies the results
that it might produce, and the type of a continuation classifies the results
that it expects to consume.  Commands do not have a type; they are just $\Ok$ to
run.  The eventual result of a command is ``escapes'' through one of its
available continuation variables.  Likewise, a program is a consistent block of
code that is capable of running, meaning that a program is a command that runs
with respect to some top-level declarations that introduce type information
(data types, type synonyms, and axioms).  The normal way to type-check a
top-level program is $\Gamma_0;\tp:\ty \vdash \pgm \Ok$, where $\Gamma_0$
specifies any primitive types and values provided by the run-time environment,
and $\tp:\ty$ is the single, top-level exit path out of the program that expects
a $\ty$ result.

Compared with System FC, more of the typing rules enjoy the
\emph{sub-formula property}, meaning that the types appearing in a premise above
the line of a rule appear somewhere below the line.  This is a natural
consequence of the sequent calculus as a logic, and was one of the primary
motivations for its original development.  The expected rules violating the
sub-formula property are the various \emph{cut} rules that cancel out arbitrary
types, given by the rules for typing commands and bindings, which effectively
perform multiple cuts simultaneously.  This is the reason that we must check
that in the command $\mcut{\binds}{\tm}{\ko}$, not only do $\tm$ and $\ko$ agree
on the same (inferred) type $\ty$, but that inferred type actually has to be of
kind $\star$.  The other interesting violators of note are:
\begin{itemize}
\item The rule for a polymorphic call-stack,
  $\app{\ty_1}{\ko} : \forall\ann{a}{\ki}.\ty_2$, which substitutes the
  specified type $\ty_1$ in for the variable $a$ in $\ty_2$ to get the type for
  the continuation.  This rule does not have the sub-formula property since
  $\ty_2[\ty_1/a]$ is a new type generated by the substitution.

  Since universal quantification is dual to existential quantification, the
  polymorphic call-stack is dual to the existential pair
  $(\tau_1,\tm) : \exists\ann{a}{\ki}.\tau_2$, and shares the same properties.
  In particular, $\app{\tau_1}{\ko}$ does not have a \emph{unique} type.  For
  example, given the continuation variable $r$ of type $Bool$, then the
  polymorphic call-stack $\app{Int}{\app{1}{\app{2}{q}}}$ can be given the types
  $\forall \ann{a}{\star}. a \to a \to Bool$,
  $\forall \ann{a}{\star}. Int \to a \to Bool$, and so on.  So following the
  bottom-up preference of a sequent calculus presentation, it is easy to type
  check a polymorphic call-stack if we already know the type of function it
  expects, but in general it is hard to guess its type.

\item The rules for pattern matching on data types almost suffers from the same
  issue as for polymorphic call-stacks, due to substitution of the choice for
  polymorphic type variables.  However, the type annotations on variables bound
  by pattern matching already specify the specialized types, so the issue is
  avoided.

  Also note that the problem with existential pairs (or in general, existential
  data structures) mentioned in the previous point is avoided on the term side.
  This is because we do not represent data structures directly, as a
  fully-applied constructor like $(\ty, \tm)$, but rather we represent them
  indirectly as a constructor evaluated with a polymorphic call-stack,
  $\cut{(,)}{\app{\ty}{\app{\tm}{q}}}$.  That means that all the troubles with
  checking existential structures is contained solely with polymorphic call
  stacks.

\item The rule for coercion continuations, $\koerce{\cn}{\ko} : \ty_1$, in which
  the type of the continuation $\ko$ is hidden by the coercion.  Interestingly,
  the typing rules for casting resembles a special sort of function application,
  both with terms in System FC and continuations in Dual System FC.
\end{itemize}

Due to the difficulty of inferring the type of a polymorphic call-stack, the
type checking algorithm for Sequent Core reveals some of its unsymmetrical bias.
On the one hand, reading the type off of a Sequent Core term (including
first-class coercions) is straightforward, since they are a subset of Core
expressions.  On the other hand, checking that a continuation has a known type
is also straightforward.  This flow of type-reading versus type-checking is
captured in the structure of a command:
\begin{itemize}
\item If the command has any local bindings, gather the types for the bound
  local variables and confirm that they are bound to well-typed terms and
  continuations.
\item Read the type off of the term in the command, confirming that the term is
  well-typed.
\item Check that the continuation in the command has the same type as the term.
\end{itemize}
This explicit flow of type information (from term to continuation) likely has
something to do with bi-directional type checking.  Intuitively, this situation
is identical to the problem with inferring the general existential tuples.  And
by duality, the solution would be to flip the flow of type information the other
way: read off the type of the \emph{continuation} that uses an existential
tuple, and then check the tuple against the inferred type.

\section{Design discussion}
\label{sec:design-discussion}

\subsection{Join points}

One of the goals of Sequent Core is to provide a good represenation for naming
join points in a program.  A \emph{join point} is a specified point in the
control flow of a program where to different possible execution paths join back
up again.  The purpose of giving a name to these join points is to avoid
excessive duplication of code and keep code size small.  Since a join point
represents the future execution path of a program, we would like to model them
as continuations, and so named join points become named continuations.

The main place where new join points are named are in the branches of a case
expression.  For example, consider the Core expression
\begin{align*}
  e &= \Case e_0 \Of \{ K_1 ~ x ~ y \to e_1; K_2 \to e_2 \}
\end{align*}
In Sequent Core, $e$ would be represented as the term $v$:
\begin{align*}
  v
  &=
  \comp{r}
    \cut
    {\comp{q}\com_0}
    {\Case\Of \{ K_1 ~ x ~ y \to \com_1; K_2 \to \com_2 \}}
\end{align*}
where the term $\comp{q}\com_0$ corresponds to the expression $e_0$, and the
commands $\com_1$ and $\com_2$ correspond to the expressions $e_1$ and $e_2$ run
in the context of the continuation $r$ which expects the end result returned by
$e$.  Now, it would be semantically correct to inline the case continuation for
$q$ inside of $\com_0$, which could open up further simplification.  For
example, suppose that $\com_0$ is:
\begin{align*}
  \com_0
  &=
  \cut{z}{\Case\Of \{ True \to \cut{K_1}{\app{5}{\app{10}{q}}}; False \to \cut{z'}{q} \}}
\end{align*}
Inlining for $q$ everywhere in $\com_0$ then corresponds to the case-of-case
transformation performed by GHC on Core expressions.

However, inlining for $q$ everywhere can duplicate the commands $\com_1$ and
$\com_2$, leading to larger code size!  We only want to inline selectively in
places where we are confident that inlining the case continuation would lead to
further simplification, but not elsewhere.  We can achieve selective inlining by
assigning the continuation to its name, $q$:
\begin{align*}
  v
  &=
  \comp{r}
  \Let q = \Case\Of \{ K_1 ~ x ~ y \to \com_1; K_2 \to \com_2 \}
  \In \com_0
\end{align*}
and then follow heuristics to inline for $q$ only in those places that matter.
In particular, we could just inline inside of the one branch in $\com_0$ which
provides $q$ with a constructed result, leading to the command:
\begin{align*}
  \com_0
  &=
  \cut{z}{\Case\Of \{ True \to \com_1[5/x,10/y]; False \to \cut{z'}{q} \}}
\end{align*}
In effect, selective inlining has eliminated case analysis on a known case by
duplicating the command $\com_1$, but not duplicating $\com_2$.

There is still a problem, though: what if the command $\com_1$ is so large that
this simplification is not worth the duplication?  In that case, we would like
to form a \emph{new} join point that gives a name to the command $\com_1$, so
that we may refer to it by name rather than duplicating the entire command.
Since $\com_1$ is an open command referring to local variables introduced by the
case analysis, we must first abstract over these local variables.  This is the
role of the polyadic form of continuation, which is able to recieve multiple
inputs before running.  In our example, we would move $\com_1$ and $\com_2$ into
a named, polyadic continuations:
\begin{gather*}
\begin{aligned}
  v
  &=
  \comp{r}
  &
    \Let
      p_1 &= \fnk{\pak{x}{y}} \com_1;
  \\
  &&
      p_2 &= \fnk{\pak{}{}} \com_2;
  \\
  &&
      q &= \Case\Of \{
          K_1 ~ x ~ y \to \cut{\pak{x}{y}}{p_1};
          K_2 \to \cut{\pak{}{}}{p_2}
          \}
  \\
  &&
    \In
    \com_0
\end{aligned}
\end{gather*}
Now, $q$ stands for a small continuation, and so it can be inlined much more
aggressively.  For example, we can perform a more lightweight inlining into
$\com_0$:
\begin{align*}
  \com_0
  &=
  \cut{z}{\Case\Of \{ True \to \cut{\pak{5}{10}}{p_1}; False \to \cut{z'}{q} \}}
\end{align*}
which avoids duplicating $\com_1$ at all.

\subsection{Polymorphic-polyadic continuations and the existential stack}

Due to the fact that some constructors to data types contain existentially
quantified types, it is important that we be able to abstract over a command
with free type variables as a polymorphic continuation.  Thus, the more general
form of the $\lambda$-continuation may introduce several type variables in
addition to term variables.  This means that the stacks which are provided to
these continuations are effectively a form of existential tuple.  Currently, we
represent these polymorphic continuations and continuations expecting multiple
inputs as an uncurried $\lambda$-abstraction.  However, there are multiple other
design choices which are possible for this purpose:
\begin{itemize}
\item Unboxed tuples: Morally, the stack $\pak{1}{\pak{2}{3}}$ for a polyadic
  continuation is the same as the unboxed tuple $(\# 1, 2, 3 \#)$.  Why then do
  we bother to introduce a new form of term, and a new type, instead of just
  using unboxed tuples for this purpose?  The reason is the above-mentioned
  existentially-quantified types, which lie outside the form of unboxed tuples
  supported by Core.  We cannot represent
  $\pak{Int}{\pak{1}{\pak{2}{3}}} : \stackty{a:\star}{a, a, a}$ as an unboxed
  tuple.

  As a minor advantage, having the special form for existential stacks lets us
  (actually, requires us to) translate them back into Core as a calling context
  rather than a value, so $\cut{\pak{1}{\pak{2}{3}}}{q}$ becomes $q ~ 1 ~ 2 ~ 3$
  instead of $q (\# 1, 2, 3 \#)$.  This results in Core expressions for join
  points looking closer to what the current GHC simplifier actually produces.

\item Curried vs uncurried: Currently, the continuations which accept
  existential stacks are written in a totally-uncurried form.  By listing all
  the paramters that abstract over a command, we are forced to spell out
  \emph{exactly} the number of inputs that are required by the continuation
  before any work can be done.

  An alternative design would be to give a curried form of continuations which
  accept existential stacks.  These would have one of two forms:
  \begin{itemize}
  \item $\fnk{\ann x \ty}k$: accept the first element of the stack, which is a
    term of type $\ty$, as $x$, then pass the rest of the stack to $k$ (i.e.,
    pop the top element of the stack as $x$ and then continue as $k$)
  \item $\fnk{\ann a \ki}k$: accept the first element of the stack, which is a
    type of kind $\ki$, as $a$, and then pass the rest of the stack to $k$
    (i.e., specialize the type variable $a$ for the continuation $k$)
  \end{itemize}
  which could be collapsed like ordinary $\lambda$-abstractions.  To model a
  unary continuation, like the continuation which accepts the ``end'' of the
  stack, we could introduce the dual to general computation abstractions.  These
  are continuations of the form $\letin{\ann x \ty}\com$ which accept an input
  named $x$ before performing some arbitrary computation $\com$, and correspond
  to the context $\Let x:\ty = \hole \in e$ in Core.  However, these
  fundamentally introduce \emph{non-strict} continuations, which is a whole can
  of worms we have been avoiding thus far.  In the end, it may be worthwhile to
  introduce these general input continuations for independent reasons, but we
  leave them out for now.

\item Case alternative: Instead of introducing a special new form of
  continuations for accepting existential stacks, we could also have added a
  form of case alternative for matching on the stack.  That way the special
  continuations $\fnk{\paks{\vect{\ann a \ki}}{\vect{\ann x \ty}}}\com$ would
  just be written as another form of case continuation
  $\Case\Of{\paks{\ann a \ki}{\ann x \ty} \to \com}$.  This unification might
  make sense following the story that existential stacks are just a sort of
  unboxed data type that are not available in Core, so it makes sense that we
  would pattern match on them.

  For now we don't fold continuations for existential stacks, but it is unclear
  if doing so would be a benefit (one less type of continuation to consider) or
  a hinderance (if they need to be treated differently in enough of the compiler
  that we effectively special case them anyways).  It is worth keeping this
  alternative design in mind, to evaluate which of the two options would pan out
  better in practice.
\end{itemize}

\subsection{Continuation-closed terms}

Currently, we make the scope and type restriction that forces all terms to never
reference free continuation variables.  This kind of restriction is necessary to
ensure that Sequent Core programs correspond directly to Core programs, and do
not introduce new kinds of control flow that would require a full
continuation-passin style (CPS) translation back into Core.  However, it is
probably sufficient to impose this scoping restriction on \emph{values} (and in
particular, $\lambda$-abstractions).  This would allow for general computation
terms ($\mu$-abstractions $\comp{q}\com$) to reference continuations in its
scope in order to produce a (continuation-closed) value.  Since general
computations cannot escape their scope, due to call-by-need semantics, this
should be fine and should not require a full CPS translation back into Core.

The scope restriction on general computation terms comes up during translation
from Core to Sequent Core.  Currently, the restriction forces the continuation
representing the context of a case expression be let-bound, rather than use a
$\mu$-abstraction.  This forces more continuations to be abstracted into the
local let bindings, which in effect introduces more named join points back in
Core world.  This is probably not bad, since we want to be introducing join
points anyway, but it is something to keep in mind.

\end{document}
